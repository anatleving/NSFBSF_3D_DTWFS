\subsection*{Overview}
Fluorescent imaging is a cornerstone of biomedical research, but its application is severely limited to superficial layers due to the strong optical scattering of biological tissue. Wavefront shaping (WFS) offers a promising way to image deep inside scattering tissue,  by physically correcting for their aberrations. However, current methods using a single planar spatial light modulator (SLM) are limited to an extremely small field-of-view, requiring slow, point-by-point scanning to image a full volume. This research aims to overcome this fundamental limitation by developing a 3D wavefront shaping system capable of correcting aberrations across a wide 3D volume. The project will target deep-tissue imaging in compact ($100-300\mu m$-sized) but highly scattering biological systems, such as early-stage embryos and organoids. The approach consists of two stages: (1) fitting a 3D aberration model of the tissue using diffraction tomography , and (2) building a novel, physical multi-conjugate correction system that can apply this 3D correction over a wide field-of-view without scanning.


\subsection*{Intellectual Merit} 
This project's intellectual merit lies in two key innovations. First, it proposes a novel method to acquire the 3D aberration model by using diffraction tomography with coherent light, which is a more robust and well-posed problem than using weak fluorescent feedback. Moreover, this can be performed using longer, NIR wavelengths where scattering is reduced and the 3D reconstruction is simpler. This 3D refractive index map will then be computationally converted to compute the necessary aberration correction for shorter, visible fluorescence wavelengths.


Second, the project will build a physical correction system that addresses the small field-of-view of planar SLMs. It will develop a multi-conjugate aberration correction system that approximates a full 3D volumetric correction by correcting aberrations at multiple depth planes simultaneously. The core innovation is a compact optical design that reuses a single planar SLM to implement multiple correction layers, overcoming the complexity and limitations of previous multi-SLM systems and enabling fast, wide-field imaging.

\subsection*{Broader impacts} 
The successful development of this system will provide access to biological events that are currently invisible, unlocking new opportunities to study early human development, improve organoid models of disease, and understand the principles of tissue regeneration in intact specimens. To maximize impact, the project will disseminate all results widely by releasing open-source software for the reconstruction and wavefront shaping algorithms, public datasets from simulations and experiments, and open-hardware designs for the multi-conjugate optical system. The project will also train the next generation of interdisciplinary scientists by organizing a new course on wavefront shaping at international conferences, hosting an interdisciplinary workshop, and actively mentoring graduate and undergraduate students from diverse backgrounds, continuing a strong existing tradition of mentorship. Finally, the PIs will develop K-12 outreach activities, including hands-on demonstrations, to spark public curiosity in optics and imaging.\\

\boldstart{Keywords:} Wavefront shaping, deep tissue imaging, fluorescent imaging, diffraction tomography.
