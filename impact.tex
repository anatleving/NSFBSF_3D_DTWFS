\vspace{-0.1cm}\section{Broader Impacts}\vspace{-0.1cm}

\boldstart{Transformative impact.}  This research has the potential to overcome the fundamental limitations of optical scattering, which currently restricts fluorescence microscopy to superficial layers. Developing a system for wide-field 3D wavefront shaping,  will provide access to biological events that are currently invisible within intact living specimens. This technology will unlock new opportunities to study complex processes in their native 3D context, such as observing how tissues pattern, how morphogenetic flows shape organs, and how early cell-fate decisions propagate in embryos and organoids. Success in this project would provide a fundamental new window into the "blueprints" of life, enabling researchers to answer key questions in developmental biology, improve organoid models of human disease, and understand the principles of tissue regeneration. This essential knowledge is a critical prerequisite for developing future therapies for a wide range of human ailments, including developmental diseases, cancer, and tissue degeneration.

%Fluorescent imaging is one of the Nobel prize inventions with the widest impact on science and is nowadays a routine procedure in any biological lab. However, at the moment it only applies to superficial tissue layers. Fluorescent sources located deeper inside the tissue are strongly aberrated due to scattering.
%The goal of this project is to significantly expand the penetration depth of fluorescent microscopes. While currently these techniques cannot see more than a few dozen microns into the tissue, the developed tools aim to image a few hundred  microns deep into the tissue. For brain imaging this will open the door to a totally new set of research questions, as the functionality of neurons in deep brain layers was barely studied in the past due to the lack of imaging tools.
%
%The project aims to study all aspects of the problem, from mathematical models and algorithms to  hardware construction. In collaboration with the lab of Prof. Hillel Adesnik in Berkeley, we also aim to demonstrate preliminary application to the imaging of live mice.
%
%
%At a broader level, this project hopes to establish a new point of convergence between computer vision, graphics, optics, imaging and biology. PI Levin comes from the computer graphics and vision world, whereas PI Waller was trained in optics and bioimaging.
%On one hand, we hope to bring advanced computational algorithms into the design of microscopes, and on the other hand  we hope to bring the challenging biological problems into the attention of researchers with strong algorithmic expertise.
%
%
%To enhance dissemination and encourage cross-pollination of ideas, the PIs will seek to publish their results in conferences and journals both within vision and graphics (ICCP, CVPR, SIGGRAPH, etc.), as well as in optics and imaging (Nature,  Science and Optica group journals, OSA and SPIE conferences). Additionally, the PIs will organize a workshop in Year Three, focusing on computational microscopy, bringing together world-experts from graphics, vision, opics, microscopy and biology. The proposed budget does not include funding for this workshop, and instead we will seek to hold it using funds from sponsorship agreements.


\boldstart{Data, software, and hardware dissemination:} To maximize impact, we will release: i)Open-source software
for diffraction tomography and wavefront shaping algorithms, ii)Public datasets including raw optical measurements, synthetic scattering simulations,
and benchmark evaluation data (shared via Zenodo and institutional repositories). iii)Hardware designs of multi-conjugate optics
and protocols including CAD files, alignment procedures, and calibration software,
enabling reproducibility by other labs.

\boldstart{Community Building and Dissemination:} We will organize a course on wavefront shaping at the IEEE
International Conference on Computational Photography (ICCP) and at optics conferences, aimed at training
the next generation of researchers at the intersection of optics, computer vision, and machine learning.
In Year 4, we will host an interdisciplinary workshop on Computational Imaging and Optimization in
Scattering Media, bringing together researchers from computer vision, optics, and biomedical imaging.

\boldstart{Mentoring and inclusive training:} Both labs are committed to providing high-quality mentorship and fostering
an inclusive research environment. All educational and training opportunities in this project --including
undergraduate research positions, graduate mentorship, workshops, and lab activities—will be open to all
interested participants regardless of background. We will actively encourage participation from individuals
across a wide range of experiences and perspectives, and will work with institutional programs that
promote engagement by students from diverse pathways into STEM. The Waller Lab has mentored 108
undergraduate students over the past 12 years. Many of these students have gone on to graduate programs
or faculty positions at top institutions including MIT, UC Berkeley, UC Davis, Cornell, and Northwestern.
We will continue this strong tradition of mentorship while ensuring equitable access.

\boldstart{K–12 outreach and optics education:} Both PIs are parents of elementary school children and are personally
invested in developing age-appropriate outreach activities to spark curiosity in optics and imaging. These activities
will build on successful programs already run by the Waller Lab, including hands-on demonstrations
and lab tours for K–12 students in the Oakland area, organized through the EECS Graduate Association. We
will expand these efforts with new modules tailored to elementary and middle school students, designed to
be engaging and accessible to all students, with the goal of broadening participation in STEM .

