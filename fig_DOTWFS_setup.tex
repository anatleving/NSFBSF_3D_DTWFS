\begin{figure*}[t!]
	\begin{center}\vspace{-0.02cm}
		\begin{tabular}{@{}c@{~~~~~~~~~~~~~~}c@{}}	
				\includegraphics[height=0.2\textwidth]{figs/system/ODT_WFS_setup2.pdf}&
				\includegraphics[height=0.2\textwidth]{figs/system/ODT_WFS_setup1.pdf}
          
		\end{tabular}
\caption{\footnotesize {\bf Setup for diffraction tomography and wavefront shaping correction:}  (a)  3D reconstruction phase: plane waves arising behind the tissue sample illuminate it and the scattered wavefronts propagating through the tissue are recorded by a front camera.   Two typical input images  under two different illumination angles are illustrated next to the front sensor illustrating significant scattering.
	(b) Wavefront shaping phase: using the recovered tissue we compute a modulation and place it on an SLM at the front end. The modulated light propagates through the tissue and focuses at a point inside the tissue, despite tissue aberration. The light emitted from the excited spot is scattered through the tissue, but corrected by the same SLM modulation before reaching the sensor. We illustrate two typical image of the fluorescent light with an without correction. With no correction we observe a wide noisy speckle pattern, but with the correction a sharp spot is visible. To better visualize the system we also display an image of the excitation light captured from the back camera. 	}
	\label{fig:dot-wfs}\vspace{-0.2cm}
	\end{center}
\end{figure*}
%	{\raisebox{0.90cm}{\rotatebox[origin=c]{90}{Main Camera }}}&