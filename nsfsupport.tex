\vspace{-0.1cm}\section{Results of prior NSF support}\vspace{-0.1cm}

\boldstart{PI Laura Waller} is currently part of the NSF Science and Technology Center on Real-Time Functional Imaging (STROBE) DMR 1548924. She collaborates with X-ray and electron microscopy researchers to do image inverse problems and algorithmic self-calibration for atomic-resolution electron tomography and X-ray phase imaging via ptychography.

\boldstart{PI Anat Levin} had a prior NSF-BSF award with PI Ioannis Gkioulekas from CMU.  (2008123/2019758) Title:  ``Computational Imaging with Speckle Correlations for Material Analysis''. 
The project was aimed to develop new tools to simulate and utilize the properties of speckles resulting from coherent scattering of light in tissue. 
The research has result in novel tools for simulating physically accurate speckles inspired by Monta-Carlo rendering algorithms. Our new simulator is orders of magnitude more efficient than previous  optics approaches, and scales to much larger scenes, this has result in a number of publications~\cite{Bar:2019,Bar:2020,single-sct-iccp-21,BarTemproralMCOptics23}. 
We also used speckle statistics to design a number of novel imaging systems, examples include: 1) Seeing Fluorescent sources deep inside the tissue~\cite{SeeThroughSubmission,Chen:22}, 2) Develop wavefront shaping systems~\cite{Dror22,DrorNatureComm24}. 3)Measuring the intrinsic optical properties of materials~\cite{Saiko22}, 4)Robust, high resolution depth sensing~\cite{Kotwal:2023:SWI,Kotwal:2023:Passive}.

\boldstart{The proposed research is not covered under any of these prior awards.} 
