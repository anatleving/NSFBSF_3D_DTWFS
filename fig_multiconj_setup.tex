\begin{figure*}[t!]\vspace{-0.2cm}
	\begin{center}
		\includegraphics[width= 0.95\textwidth]{figs/system/multiconj.pdf}
	\end{center}\vspace{-0.3cm}
	\caption{{\bf{Setups for multi-conjugate confocal correction:}} (a,b) illustrate two possible ways to implement multi-conjugate corrections. We illustrate two light paths emerging from two different points inside the tissue, which pass through relay systems with 3 SLMs on their way to the sensor.  The SLMs are placed at different defocus planes, namely, each of them is conjugate to a different plane inside the tissue volume, so they each can correct aberrations in a different depth. For ease of visualization, the figure marks 3 depth layers in different colors, and follows the copies of these layers through the relay systems. In (a) we visualize an order preserving setup where the correction of the first tissue plane $p_1$ is applied first, and in (b) an order reverting setup, where the correction of the last plane is applied first. This arrangement  requires additional relays, however we show that its more effective in inverting the physical aberration process.
		In (c) we compare the two setups. We use a collection of scattered wavefronts measured in the lab through a real tissue slice and attempt to explain them using multi-slice models matching the order preserving and order reverting setups. We plot the fitting error as a function of the number of correction layers and see that the order-reverting setup provides a significantly better fit.    }\label{fig:setup_multiconj}
\end{figure*}

