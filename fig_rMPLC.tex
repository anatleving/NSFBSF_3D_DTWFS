\begin{figure*}[t!]\vspace{-0.2cm}
	\begin{center}
		\begin{tabular}{@{}c@{}}
		\begin{tabular}{@{}c@{}c@{}}
			\includegraphics[width= 0.55\textwidth]{figs/system/setup_rMPLC_xz.pdf}&
				\includegraphics[width= 0.45\textwidth]{figs/system/setup_rMPLC_xy.pdf}
		\end{tabular}\\
	\begin{tabular}{@{}c@{}}
			\includegraphics[width= 0.65\textwidth]{figs/system/setup_rMPLC_defocus.pdf}
	\end{tabular}
		\end{tabular}
		
	\end{center}\vspace{-0.3cm}
	\caption{{\bf{Setup for a relayed  multi-plane-light conversion:}} We design a multi-conjugate correction system that reuses a single SLM and a single relay system to implement multiple correction layers. (a) A side view of the optics. Light emerging from a point inside the tissue is relayed via a standard objective and tube lens. Then a mirror at $45^o$ angle reflect it into the correction unit where it passes multiple bounces until reaching the exit mirror, from which it is relayed to the sensor. The correction unit consists of a SLM, a lens and a mirror surface behind it, which together form a reflective 4f relay system. Note that different bounces intersect at different positions on the SLM plane so different correction pasterns can be displayed. There is also a small defocus between the bounces, the first one is wide and the last one is narrow so that each of them corresponds to a different depth plane of the tissue. (b) Top view of the SLM plane and the places where the different bounces intersect it (this time more bounces are enclosed), plus an enlarged side view of the tissue. Each bounce is designed to be an image of a different plane inside the tissue and corresponding planes are encoded with the same color. The bounce corresponding to $p_5$ is wide and the one corresponding to $p_1$ is narrow, as the physical cone intersection of these planes in the tissue has a different size. (c-e) illustrate how different defocus levels are achieved. If the SLM is placed exactly at the focal plane of the lens as in (c) a perfect 4f is obtained, and light emerging from a point would reflect though the relay and return exactly to the same point. If the SLM is shifted a bit closer to the lens the travel distance is shortened and returning rays converge closer to the SLM plane as in (d). A successive bounce illustrated in (e) converges even closer.
	 }\label{fig:setup_rMPLC}
\end{figure*}

